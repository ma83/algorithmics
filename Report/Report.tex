\documentclass[a4paper]{article}

\usepackage{amsmath}
\usepackage{amssymb}
\usepackage{amsfonts}
\usepackage{scrpage2}
\usepackage{graphicx}
\usepackage[T1]{fontenc}
\usepackage[utf8]{inputenc}
\usepackage[inner=2.5 cm,outer=2.5 cm,top=2.5 cm,bottom=2.5 cm]{geometry}
\usepackage[small, center]{caption}
\usepackage{setspace}
%\usepackage[USenglish]{babel}
%\usepackage[babel,german=guillemets]{csquotes}
\usepackage[backend=biber, style=authortitle-comp]{biblatex}
%\usepackage[sort]{natbib}
%\bibliographystyle{apalike}
\bibliography{biblio2}

%For Sourcecode
\usepackage{listings}



\begin{document}

\pagestyle{scrheadings}
\ohead{Algorithmics, Prog. Ex., WS 14/15} 
\ihead{\pagemark} 
\pagenumbering{roman}

\author{Stefan  \and Robin Melán}
\title{Time-Constrained Bipartite Vehicle Routing Problem (TCBVRP)}
\maketitle
\vspace{0.5cm}

\section*{Problem description}
Our task in the TCBVRP\footnote{Programming Exercise / ILOG CPLEX Tutorial, Benjamin Biesinger, Bin Hu, Gunther R. Raidl, Algorithms and Data Structures Group Institute of Computer Graphics and Algorithms, Vienna University of Technology, VU Algorithmics, WS 2014/15} is to supply a given number of demand nodes by minimizing the unit of time by the number of vehicles used. A vehicle tour starts at the depot node and after an alternating ordering of getting first to a supply and then to a demand node the tour finishes at the depot again. The time capacity of every crossed edge is added up and has a given constraint of time limit per vehicle that must not be exceeded. Aim is to minimize the sum of time periods of all used vehicles in the graph.\\
For this problem we developed the following integer programming formulations, which we solved with CPLEX. 

\subsection*{Variables}
\begin{align*}
n &: \text{Total number of Nodes} \\
m &: \text{Total number of Nodes} \\
T &: \text{Timelimit for one vehicle} \\
N &: \text{Set of all nodes} \\
S &: \text{Set of supply nodes} \\
D &: \text{Set of demand nodes} \\
V &: \text{Set of vehicles} \\
\end{align*}

\subsection*{Declaration of the decision variables}
\begin{equation*}
x_{ij}^k : 
	\begin{cases}
	1 & \text{if edge from i to j is crossed by vehicle k} \\
	0 & \text{otherwise} \\
	\end{cases}
\end{equation*}

\subsection*{Objective Function}
Minimize the total time of all used vehicles
\begin{equation}
min. \quad \sum_{k=1}^m \sum_{i=1}^n  \sum_{j=1}^n x_{ij}^k * d_{ij}
\end{equation}

\subsection*{Constraints}
Für jedes Fahrzeug muss das timelimit T eingehalten werden [TODO short description]
\begin{equation}
\sum_{i=1}^n  \sum_{j=1}^n x_{ij}^k * d_{ij} \leq T		\quad \quad		\forall k \in V 
\end{equation}

Jeder Demand muss von einem Supply aus angefahren werden [TODO short description]
\begin{equation}
\sum_{i \in S} \sum_{k \in V} x_{ij}^k = 1		\quad 
\quad		\forall j \in D
\end{equation}

Supplies dürfen maximal einmal angefahren werden - von Depot aus oder Demand [TODO short description]
\begin{equation}
\sum_{i \in D} \sum_{k \in V} x_{ij}^k \leq 1		\quad 
\quad		\forall j \in S
\end{equation}

Die Ausgehenden Kanten vom Depot müssen größer 1 sein [TODO short description]
\begin{equation}
\sum_{k \in V} \sum_{j \in S} x_{1j}^k \geq 1		\quad 
\end{equation}

Ein Fahrzeug darf nicht 2mal losfahren vom Depot [TODO short description]
\begin{equation}
\sum_{j \in S} x_{1j}^k \leq 1		\quad 
\quad		\forall k \in V 
\end{equation}

Die Anzahl der Fahrzeuge die in einen Node reinfahren muss gleich sein der Summe die rausfahren [TODO short description]
\begin{equation}
\sum_{j=1}^n x_{ij}^k = \sum_{j=1}^n x_{ji}^k		\quad 
\quad		\forall i \in N 
\quad		\forall k \in V 
\end{equation}

Man darf vom Depot nur zu Supplies fahren [TODO short description]
\begin{equation}
x_{1j}^k = 0		\quad 
\quad		\forall j \in D
\quad		\forall k \in V 
\end{equation}

Man darf von Demand nur zu Supplie oder Depot fahren [TODO short description]
\begin{equation}
\sum_{i}^n x_{ij}^k = 0		\quad 
\quad		\forall j \in D
\quad		\forall k \in V 
\end{equation}

Man darf von Supply nur zu Demand fahren [TODO short description]
\begin{equation}
\sum_{j,j \not\in D}^n x_{ij}^k = 0		\quad 
\quad		\forall i \in S
\quad		\forall k \in V 
\end{equation}

Alle Demand Nodes müssen 1 mal angefahren werden [TODO short description]
\begin{equation}
\sum_{k}^m \sum_{j}^n x_{ji}^k = 1		\quad 
\quad		\forall i \in D 
\end{equation}


\section*{Single Commodity Flow (SCF)}
\subsection*{Declaration of the decision variables}
\begin{equation*}
f_{ij}^k : 
	\begin{cases}
	\text{represents the amount of "flow" on arc (i,j)}
	\end{cases}
\end{equation*}

\subsection*{Sub-tour Elimination Constraint}
a[TODO short description]
\begin{equation}
\sum_{j, j\neq 1} f_{1j}^k = n - 1
\quad \forall k \in V 
\end{equation}
[TODO short description]
\begin{equation}
\sum_{i, i\neq j} f_{ij}^k - \sum_{h, h\neq j} f_{jh}^k \leq 1 \quad 
\quad \forall j \in N \setminus {1}
\quad \forall k \in V 
\end{equation}
[TODO short description]
\begin{equation}
0 \leq f_{ij} \leq (n - 1) * x_{ij}^k \quad 
\quad \forall i,j \in N, i \neq j
\quad \forall k \in V 
\end{equation}


\section*{Multi Commodity Flow (MCF)}
\subsection*{Declaration of the decision variables}
\begin{equation*}
f_{ij}^{kl} : 
	\begin{cases}
	\text{represent the amount of "flow" on arc (i,j) for commodity l} \\
	\end{cases}
\end{equation*}

\subsection*{Sub-tour Elimination Constraint}
Kante muss verwendet werden [TODO short description]
\begin{equation}
f_{ij}^{kl} \leq x_{ij}^k		 \quad 
\quad \forall l \in N \setminus {1}
\quad \forall i \in N 
\quad \forall j \in N , i \neq j
\quad \forall k \in V 
\end{equation}

Der Ausgesendete Commodity muss bei der bestimmten l == 1 sein [TODO short description]
\begin{equation}
\sum_{i, i \neq l}^n f_{il}^{kl} = 1		 \quad 
\quad \forall l \in N \setminus {1}
\quad \forall k \in V 
\end{equation}

Der Depotnode sendet an jede Commodity max 1 aus [TODO short description]
\begin{equation}
\sum_{i, i \neq 1}^n f_{1i}^{kl} = 1		 \quad 
\quad \forall l \in N \setminus {1}
\quad \forall k \in V 
\end{equation}

Jede Commodity muss verbraucht werden [TODO short description]
\begin{equation}
\sum_{i, i \neq 1}^n f_{i1}^{kl} = 0		 \quad 
\quad \forall l \in N \setminus {1}
\quad \forall k \in V 
\end{equation}

[vorletzter constraint nicht ganz klar woher? Ich glaube der gehört weg Bitte mit c++ file kontrollieren]
\begin{equation}
\sum_{i}^n f_{li}^{kl} = 0		 \quad 
\quad \forall l \in N 
\quad \forall k \in V 
\end{equation}

letzter constraint [TODO short description]
\begin{equation}
\sum_{i, i \neq j}^n f_{ij}^{kl} - \sum_{i, i \neq j}^n f_{ji}^{kl} = 0		 \quad 
\quad \forall l \in N \setminus {1}
\quad \forall j \in N , j \neq l
\quad \forall k \in V 
\end{equation}


\section*{Miller-Tucker-Zemlin subtour elimination constraints (MTZ)}\subsection*{Declaration of the decision variables}
\begin{equation*}
r_{ik} : 
	\begin{cases}
	\text{represents the rank of node to describe the flow}
	\end{cases}
\end{equation*}

\subsection*{Sub-tour Elimination Constraint}
a[TODO short description]
\begin{equation}
r_{ik} + n * x_{ij}^k \leq r_{ik} + (n - 1) \quad 
\quad \forall i \in N \setminus {1}
\quad \forall j \in N \setminus {1} , i \neq j
\quad \forall k \in V 
\end{equation}

\section*{Results and Discussion}


\begin{table}[h!]
	\centering
	\caption{Solutions with the Single Commodity Flow Constraints}
		\begin{tabular}{|c|r|r||c|c|c|}
			\hline\
		Nodes & Time & Vehicles	& Objective Function & \# of Branch \& Bound & Runtime\\
   		\multicolumn{1}{|c|}{[n]} & \multicolumn{1}{c|}{[sec]} & \multicolumn{1}{c||}{[m]} & \multicolumn{1}{c|}{Values} & \multicolumn{1}{c|}{Nodes} & \multicolumn{1}{c|}{[sec]} \\
			\hline 
				10	&	240	&	2	&	347		&		&		\\
				10	&	480	&	2	&	308		&		&		\\
				10	&	240	&	2	&	275		&		&		\\
				10	&	480	&	2	&	252		&		&		\\
				20	&	240	&	3	&	533		&		&		\\
				20	&	480	&	2	&	517		&		&		\\
				20	&	240	&	4	&	594		&		&		\\
				20	&	480	&	2	&	566		&		&		\\
				30	&	240	&	4	&	763		&		&		\\
				30	&	480	&	2	&	717		&		&		\\
				30	&	240	&	4	&	896		&		&		\\
				30	&	480	&	2	&	852		&		&		\\
				60	&	360	&	6	&	1699	&		&		\\
				60	&	480	&	4	&	1671	&		&		\\
				90	&	480	&	8	&	2031	&		&		\\
				120	&	480	&	8	&	2663	&		&		\\
				180	&	720	&	10	&	5782	&		&		\\
				
			\hline 
		\end{tabular}
	\label{tab:scf}
\end{table}

\begin{table}[h!]
	\centering
	\caption{Solutions with the Multi Commodity Flow Conservation}
		\begin{tabular}{|c|r|r||c|c|c|}
			\hline\
		Nodes & Time & Vehicles	& Objective Function & \# of Branch \& Bound & Runtime\\
   		\multicolumn{1}{|c|}{[n]} & \multicolumn{1}{c|}{[sec]} & \multicolumn{1}{c||}{[m]} & \multicolumn{1}{c|}{Values} & \multicolumn{1}{c|}{Nodes} & \multicolumn{1}{c|}{[sec]} \\
			\hline 
				10	&	240	&	2	&	347		&		&		\\
				10	&	480	&	2	&	308		&		&		\\
				10	&	240	&	2	&	275		&		&		\\
				10	&	480	&	2	&	252		&		&		\\
				20	&	240	&	3	&	533		&		&		\\
				20	&	480	&	2	&	517		&		&		\\
				20	&	240	&	4	&	594		&		&		\\
				20	&	480	&	2	&	566		&		&		\\
				30	&	240	&	4	&	763		&		&		\\
				30	&	480	&	2	&	717		&		&		\\
				30	&	240	&	4	&	896		&		&		\\
				30	&	480	&	2	&	852		&		&		\\
				60	&	360	&	6	&	1699	&		&		\\
				60	&	480	&	4	&	1671	&		&		\\
				90	&	480	&	8	&	2031	&		&		\\
				120	&	480	&	8	&	2663	&		&		\\
				180	&	720	&	10	&	5782	&		&		\\
				
			\hline 
		\end{tabular}
	\label{tab:scf}
\end{table}

\begin{table}[h!]
	\centering
	\caption{Solutions with the Miller-Tucker-Zemlin subtour elimination constraints}
		\begin{tabular}{|c|r|r||c|c|c|}
			\hline\
		Nodes & Time & Vehicles	& Objective Function & \# of Branch \& Bound & Runtime\\
   		\multicolumn{1}{|c|}{[n]} & \multicolumn{1}{c|}{[sec]} & \multicolumn{1}{c||}{[m]} & \multicolumn{1}{c|}{Values} & \multicolumn{1}{c|}{Nodes} & \multicolumn{1}{c|}{[sec]} \\
			\hline 
				10	&	240	&	2	&	347		&		&		\\
				10	&	480	&	2	&	308		&		&		\\
				10	&	240	&	2	&	275		&		&		\\
				10	&	480	&	2	&	252		&		&		\\
				20	&	240	&	3	&	533		&		&		\\
				20	&	480	&	2	&	517		&		&		\\
				20	&	240	&	4	&	594		&		&		\\
				20	&	480	&	2	&	566		&		&		\\
				30	&	240	&	4	&	763		&		&		\\
				30	&	480	&	2	&	717		&		&		\\
				30	&	240	&	4	&	896		&		&		\\
				30	&	480	&	2	&	852		&		&		\\
				60	&	360	&	6	&	1699	&		&		\\
				60	&	480	&	4	&	1671	&		&		\\
				90	&	480	&	8	&	2031	&		&		\\
				120	&	480	&	8	&	2663	&		&		\\
				180	&	720	&	10	&	5782	&		&		\\
				
			\hline 
		\end{tabular}
	\label{tab:scf}
\end{table}

%\vspace{0.5cm} 
 	     	

\end{document}
